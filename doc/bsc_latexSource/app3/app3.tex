\XeTeXinputencoding cp1250
% ********** Appendix 3 **********
\chapter{Codul aplica�iei (par�ial)}
\label{sec:appendix3}

\texttt{Carata Lucian, calucian[at]yahoo.co.uk\\
Copyright (C) 2009, Carata Lucian\\
Acest program este liber; �l pute�i redistribui �i/sau modifica �n conformitate cu termenii Licen�ei Publice Generale GNU, a�a cum este publicat� de c�tre Free Software Foundation; fie versiunea 3 a Licen�ei, fie (la latitudinea dumneavoastr�) orice versiune ulterioar�.\\
Acest program este distribuit cu speran�a c� va fi util, dar F�R� NICI O GARAN�IE; f�r� macar garan�ia implicit� de vandabilitate sau CONFORMITATE UNUI ANUMIT SCOP. A se vedea Licen�a Public� General� GNU pentru detalii.\\
Ar trebui s� fi primit o copie a Licen�ei Publice Generale GNU �mpreun� cu acest program.\\
�n caz contrar, consulta�i http://www.gnu.org/licenses/.
}

\begin{framed}
\texttt{IImageTransform.h: General interface for Image transformations}
\begin{lstlisting}
#ifndef _IIMAGETRANSFORM_H_
#define _IIMAGETRANSFORM_H_

#ifndef CYTRUSALGLIB_API
	#ifdef CYTRUSALGLIB_EXPORTS
		#define CYTRUSALGLIB_API __declspec(dllexport)
	#else
		#define CYTRUSALGLIB_API __declspec(dllimport)
	#endif
#endif

namespace cytrus{
	namespace alg{
		class CYTRUSALGLIB_API IImageTransform{
		public:
			template <typename SrcView, typename DstView>
			void applyTransform(SrcView& src, DstView& dst);

			virtual ~IImageTransform(){}
		};
	}
}

#endif
\end{lstlisting}
\end{framed} 
\begin{framed}
\texttt{IDescriptor.h : An interface that defines the contract for describing the points of interest in an image.}
\begin{lstlisting}
#ifndef _IDESCRIPTOR_H_
#define _IDESCRIPTOR_H_

#ifndef CYTRUSALGLIB_API
	#ifdef CYTRUSALGLIB_EXPORTS
		#define CYTRUSALGLIB_API __declspec(dllexport)
	#else
		#define CYTRUSALGLIB_API __declspec(dllimport)
	#endif
#endif

namespace cytrus{
	namespace alg{
		class CYTRUSALGLIB_API IDescriptor{
			public:
			virtual ~IDescriptor(){}
		};
	}
}

#endif
\end{lstlisting}
\end{framed}
\begin{framed}
\texttt{ILocator.h : An interface that defines the contract for locating the points of interest in an image.}
\begin{lstlisting}
#ifndef _ILOCATOR_H_
#define _ILOCATOR_H_

#ifndef CYTRUSALGLIB_API
	#ifdef CYTRUSALGLIB_EXPORTS
		#define CYTRUSALGLIB_API __declspec(dllexport)
	#else
		#define CYTRUSALGLIB_API __declspec(dllimport)
	#endif
#endif

#include <vector>
#include "poi.h"

namespace cytrus{
	namespace alg{
		class CYTRUSALGLIB_API ILocator{
			public:
				virtual void locatePOIInImage(std::vector<Poi>& iPts_out)=0;
				virtual ~ILocator(){}
		};
	}
}

#endif
\end{lstlisting}
\end{framed}  
\begin{framed}
\texttt{IntegralImageTransform.h: Transforms a given image into its integral representation}
\begin{lstlisting}
#ifndef _INTEGRALIMAGETRANSFORM_H_
#define _INTEGRALIMAGETRANSFORM_H_


#include "IImageTransform.h"
#include <boost/gil/gil_all.hpp>

using namespace boost::gil;

namespace cytrus{
	namespace alg{
		class IntegralImageTransform : public IImageTransform{
		public:

			// SrcView and DstView must both be GIL Image Views. The following conditions must be met:
			// ImageViewConcept<SrcView> >();
			// MutableImageViewConcept<DstView> >();
			// ColorSpacesCompatibleConcept<color_space_type<SrcView>::type, color_space_type<DstView>::type> >();
			template <typename SrcView, typename DstView>
			static void applyTransform(SrcView& src, DstView& dst);

			// IntegralImageView must be a GIL Image View. The following conditions must be met:
			// ImageViewConcept<SrcView> >();
			// The current implementation assumes the view has only one channel
			template <typename IntegralImageView>
			static float boxFilter(IntegralImageView* src, int xSt, int ySt, int boxHeight, int boxWidth);
		};
		
	}
}

#include "IntegralImageTransform.hpp"

#endif
\end{lstlisting}
\end{framed}
\begin{framed}
\texttt{IImageSource.h : Image Source Interface}
\begin{lstlisting}
#ifndef _IIMAGESOURCE_H_
#define _IIMAGESOURCE_H_

#ifndef CYTRUSALGLIB_API
	#define CYTRUSALGLIB_API
#endif

#include <set>
#include "IImageConsumer.h"

namespace cytrus{
	namespace cameraHAL{
		
		// Abstract base class for all image sources.
		// Defines the functions and members for registering and removing image consumers (IImageConsumer)
		// Register your image consumer by calling registerImageConsumer, and the processImage function
		// will be called by the image source when an image is available.
		// Current available implementations:
		//	cytrus::cameraHal::DirectShowCameraSource - win32/directx only
		//	cytrus::cameraHal::ImageFromFileSource
		// Info for subclassing:
		// If your source provides external consumer notification (like function callbacks provided by a
		// SO-specific image aquisition system), you should implement those in separate (static?) functions. 
		// In derived classes, implement the four pure virtual functions,
		//	notifyConsumers() - for manual consumer notification
		//	getImageSize() - for providing source-dependent image size information
		class CYTRUSALGLIB_API IImageSource{
			protected: 
				std::set<IImageConsumer*> consumers;
				bool _sourceIsStarted;
			public:

				virtual ~IImageSource();

				//
				// This function manually notifies all the registered consumers.
				// The implementation usually calls IImageConsumer.processImage, passing the source's image
				// as a parameter.
				//
				virtual void notifyConsumers()=0;

				//
				// This function manually notifies a specific registered consumer.
				// The implementation usually calls IImageConsumer.processImage, passing the source's image
				// as a parameter.
				//
				virtual void notifyConsumer(int consumerIndex)=0;

				//
				// Returns the size <width,height> of the images that this
				// source currently provides.
				//
				virtual std::pair<int,int> getImageSize()=0;

				//
				// Starts the capture on the specific image source. This function will be called
				// after configuring the image source. 
				// Implementation examples:
				//	For a static single-image source, calling the startCapture function should read the
				// image from disk and then notify all the consumers.
				//	For a stream-type image source, calling the startCapture should determine periodic
				// notifications to the consumers, as the frames come from the capture device.
				//
				//
				virtual void startCapture()=0;

				//
				// Stops the capture process on the specific image source.
				//
				virtual void stopCapture()=0;

				//
				// Default implementation does nothing. Subclasses might overwrite this to account for the cases when
				// the size of the image provided by the source changes. Maybe it's a camera source and the user
				// connected a second camera that provides another resolution (for example)
				//
				virtual void notifySizeChange(){};

				//
				// Registers an ImageConsumer with this source. 
				// The registered ImageConsumers will be notified when new images are produced by the source,
				// and the image will be sent to them for processing. 
				// 
				// returns true if the consumer could be registered correctly
				//         false if the consumer already exists in the list;
				//               another possibility is that the source is already started
				//               for mentaining consistency, adding consumers "on the fly" is not supported
				//
				bool registerImageConsumer(IImageConsumer *c);

				//
				// Remove Image Consumer from the consumers that are waiting for this image source.
				// returns true if the consumer could be removed correctly
				//         false if the consumer did not previously exist in the list;
				//               another possibility is that the source is already started
				//               for mentaining consistency, removing consumers "on the fly" is not supported
				//
				bool removeImageConsumer(IImageConsumer *c);
		};
	
	}
}

#endif
\end{lstlisting}
\end{framed}
\begin{framed}
\texttt{ImageConsumer.h : Image Consumer Interface}
\begin{lstlisting}
#ifndef _IIMAGECONSUMER_H_
#define _IIMAGECONSUMER_H_

#ifndef CYTRUSALGLIB_API
	#define CYTRUSALGLIB_API
#endif

namespace cytrus{
	namespace cameraHAL{

		class IImageSource;

		//
		// Abstract base class for ImageConsumers
		// All the classes that make use of image sources should implement this, so that they can be notified
		// with images when those are available at the source.
		//
		class CYTRUSALGLIB_API IImageConsumer{
			protected:
				IImageSource* _imgSource;
			public: 
				bool removeFromSourceOnDestroy;
				int consumerIndex; // for identifying multiple consumers
				IImageConsumer(IImageSource* imgSource, int index);
				virtual ~IImageConsumer();
				//
				// This function is called by the image source when the image data is ready
				// (has been loaded from disk, has been recieved from a camera etc.). 
				//
				virtual void processImage(unsigned long dwSize, unsigned char* pbData)=0;

				//
				// This function is called by the image source when the size of the image
				// changes.
				//
				virtual void onSourceSizeChange()=0;
		};
	
	}
}

#endif
\end{lstlisting}
\end{framed}
\begin{framed}
\texttt{FileImageSource.h : File Image Source. Implements IImageSource}
\begin{lstlisting}
#ifndef _FILEIMAGESOURCE_H_
#define _FILEIMAGESOURCE_H_

#ifndef CYTRUSALGLIB_API
	#ifdef CYTRUSALGLIB_EXPORTS
		#define CYTRUSALGLIB_API __declspec(dllexport)
	#else
		#define CYTRUSALGLIB_API __declspec(dllimport)
	#endif
#endif

#include "IImageSource.h"
#include <boost/gil/gil_all.hpp>


namespace cytrus{
	namespace cameraHAL{

		//
		// FileImageSource is a source that reads images from files. 
		// <Limitation> Currently, the implementation only accepts .jpg files
		//
		class CYTRUSALGLIB_API FileImageSource: public IImageSource{
			private:
				unsigned long imageDataSize;
				unsigned char* imageData;

				const char* _path;
				boost::gil::rgb8_image_t* imageFile;
				boost::gil::rgb8_view_t* imageFileView;

			public:

				int width, height;
				
				FileImageSource();
				virtual ~FileImageSource();

				virtual void notifySizeChange(){};
				virtual void notifyConsumers();
				virtual void notifyConsumer(int consumerIndex);


				virtual void startCapture();
				virtual void stopCapture();
				virtual void setPath(const char* path);
				virtual std::pair<int,int> getImageSize();

		};
	
	}
}


#endif
\end{lstlisting}
\end{framed}
\begin{framed}
\texttt{DirectShowCameraSource.h : Camera Image Source. Implements IImageSource}
\begin{lstlisting}
#ifndef _CAMERASOURCE_H_
#define _CAMERASOURCE_H_

#ifndef CYTRUSALGLIB_API
	#ifdef CYTRUSALGLIB_EXPORTS
		#define CYTRUSALGLIB_API __declspec(dllexport)
	#else
		#define CYTRUSALGLIB_API __declspec(dllimport)
	#endif
#endif

#include <list>
#include "WebCamLib.h"
#include "IImageSource.h"

#define NO_CAMERA -256

namespace cytrus{
	namespace cameraHAL{

		#ifdef WIN32
		//
		// DirectShowCameraSource is an image source aquiring images from camera devices
		// (connected by usb, firewire etc) through DirectShow/DirectX
		// As such, this is a Windows-specific implementation.
		// Not threadsafe (yet)
		//
		typedef void (__stdcall *NewImageAvailableCallback)(void);

		class CYTRUSALGLIB_API DirectShowCameraSource: public IImageSource{
			// TODO: add thread safetyness to singleton & other members.
			private:
				static DirectShowCameraSource* instance;  // singleton instance
				static NewImageAvailableCallback signalNewImageAvailable;
				IUnknown** deviceHandle;

				static DWORD imageDataSize;
				static BYTE* imageData;
				
				// current capture device spec:
				int _currentCamera;
				
				//
				
				DirectShowCameraSource();
				virtual ~DirectShowCameraSource();
				void getCameraList();

			public:

				int width, height;
				int _nrAvailableCameras;
				std::list<char*> _availableCameras;


				static DirectShowCameraSource* getCameraInstance(NewImageAvailableCallback callback);
				std::list<char*> getAvailableCameras(bool refresh=true);
				void setActiveCamera(int cIndex);
				void displayCameraPropertiesDialog(HWND hwnd);

				virtual void notifySizeChange(){};
				virtual void notifyConsumers();
				virtual void notifyConsumer(int consumerIndex);

				virtual void startCapture();
				virtual void stopCapture();
				virtual std::pair<int,int> getImageSize();

				static void __stdcall callbackFunc(DWORD dwSize, BYTE* pbData);
		};
		#endif
	
	}
}


#endif
\end{lstlisting}
\end{framed}
\begin{framed}
\texttt{IPOIAlgorithm.h : An interface that inherits IImageConsumer, and enforces a contract for all the algorithms that work on Points of Interest (POI), for object detection ( Locator + Descriptor )}
\begin{lstlisting}
#ifndef _IPOIALGORITHM_H_
#define _IPOIALGORITHM_H_

#ifndef CYTRUSALGLIB_API
	#ifdef CYTRUSALGLIB_EXPORTS
		#define CYTRUSALGLIB_API __declspec(dllexport)
	#else
		#define CYTRUSALGLIB_API __declspec(dllimport)
	#endif
#endif

#include "IImageConsumer.h"
#include "ILocator.h"
#include "IDescriptor.h"
#include <list>

using namespace cytrus::cameraHAL;
using namespace cytrus::alg;

namespace cytrus{
	namespace alg{
		typedef void (__stdcall *POIAlgResult)(unsigned long dwSize, unsigned char* pbData, int index);
		class CYTRUSALGLIB_API IPOIAlgorithm : public IImageConsumer{
			protected:
				ILocator *_poiLoc;
				IDescriptor *_poiDescr;
				POIAlgResult _outputAlgResult;
				std::vector<Poi> iPts;

				std::list<std::pair<char*,int>*>* _outputModes;
				volatile int _currentOutputMode;
			public:
				IPOIAlgorithm(IImageSource* imgSrc, ILocator* poiLocator, IDescriptor* poiDescriptor, POIAlgResult outputFunc, int index);
				virtual ~IPOIAlgorithm();

				//This function should be overriden to process the image data from the source.
				//It will be automatically called by the image source, when the algorithm runs.
				//If one has to display the results of the algorithm to an outside source,
				//the implementation of this function can call _outputAlgResult with appropiate data.
				virtual void processImage(unsigned long dwSize, unsigned char* pbData)=0;

				//Allows for processing the image at a smaller size than captured
				//returns true if the processing size could be set (is valid, processing size<image size)
				virtual bool setProcessingSize(int newWidth, int newHeight)=0;
				
				//Actions to take when the image from the source changes size
				virtual void onSourceSizeChange()=0;
				
				//Manage algorithm Output modes
				virtual void setOutputMode(int mode);
				virtual int getCurrentOutputMode();
				virtual std::list<std::pair<char*,int>*>* getOutputModes();
				
				//Runs the POI Algorithm
				void run();

				std::vector<Poi> getPoiResult();
		};
	}
}

#endif

\end{lstlisting}
\end{framed}
\begin{framed}
\texttt{FastHessianLocator.h : Locating POI's using the determinant of the Hessian matrix.
Some parts are adapted from Bay's SURF implementation/notes:C. Evans, Research Into Robust Visual Features, MSc University of Bristol, 2008. }
\begin{lstlisting}
#ifndef _FASTHESSIANLOCATOR_H_
#define _FASTHESSIANLOCATOR_H_

#ifndef CYTRUSALGLIB_API
    #ifdef CYTRUSALGLIB_EXPORTS
        #define CYTRUSALGLIB_API __declspec(dllexport)
    #else
        #define CYTRUSALGLIB_API __declspec(dllimport)
    #endif
#endif

#include <vector>
#include <iostream>
#include "poi.h"
#include "ILocator.h"

namespace cytrus{
    namespace alg{

        static const int OCTAVES = 3;
        static const int INTERVALS = 4;
        static const float THRES = 0.002f;
        static const int INT_SAMPLE = 2;

        template <typename IntegralImageView>
        class FastHessianLocator:public ILocator{

            private:

                void buildDet(); // this fills the hessianDet with values calculated
                         // for pixels in the image (sampled)
                         
                //! Interpolation functions - adapted from Lowe's SIFT implementation
                // (Adapted from Bay, Evans - SURF, the above coment is from their
                //  implementation)
                void interpolateExtremum(std::vector<Poi>& iPts_out, int octv, int intvl, int r, int c);
                void interpolateStep(int octv, int intvl, int r, int c, double* xi, double* xr, double* xc );
                double* deriv3D( int octv, int intvl, int r, int c );
                double* hessian3D(int octv, int intvl, int r, int c );
                // Non-maximum supression
                int isExtremum(int octv, int intvl, int column, int row); 
                
                //utility
                inline int getLaplacianSign(int o, int i, int c, int r);
                inline float getHessian(int octave, int interval, int column, int row);
            protected:

            IntegralImageView* _img;
            int i_width, i_height;

            int _octaves;
            int _intervals;
            int _sampling;
            int _threshold;

            float* hessianDet; // array that contains

            public:
                FastHessianLocator(IntegralImageView& intImg, 
                                   const int octaves = OCTAVES, 
                                   const int intervals = INTERVALS, 
                                   const int sampling = INT_SAMPLE, 
                                   const float threshold = THRES);

                FastHessianLocator(const int octaves = OCTAVES, 
                                   const int intervals = INTERVALS, 
                                   const int sampling = INT_SAMPLE, 
                                   const float threshold = THRES);

				virtual ~FastHessianLocator();

                void setSourceIntegralImg(IntegralImageView& intImg);
                void setParameters(const int octaves, 
                                   const int intervals, 
                                   const int sampling , 
                                   const float threshold);
                
                // Outputs in the given vector the interest points detected in the
                // image, having only the location and scale information
                // computed for them. 
                virtual void locatePOIInImage(std::vector<Poi>& iPts_out);
        };
    }
}

#include "FastHessianLocator.hpp"

#endif
\end{lstlisting}
\end{framed}
\begin{framed}
\texttt{ObjectPoiStorage.h : Stores the interest points for objects to be detected}
\begin{lstlisting}
#ifndef _OBJECTPOISTORAGE_H_
#define _OBJECTPOISTORAGE_H_

#ifndef CYTRUSALGLIB_API
	#ifdef CYTRUSALGLIB_EXPORTS
		#define CYTRUSALGLIB_API __declspec(dllexport)
	#else
		#define CYTRUSALGLIB_API __declspec(dllimport)
	#endif
#endif
#define FLT_MAX         3.402823466e+38F        /* max value */

#include "IPOIAlgorithm.h"

namespace cytrus{
	namespace alg{

		class CYTRUSALGLIB_API ObjectPoiStorage{
			// TODO: add thread safetyness to singleton & other members.
			private:
				static ObjectPoiStorage* instance;  // singleton instance
				static std::vector<std::vector<Poi>*> _registeredObjects;
				
				ObjectPoiStorage();
				virtual ~ObjectPoiStorage();

				static float comparePOIs(Poi& p1, Poi& p2);

			public:

				static ObjectPoiStorage* getInstance();
				static int registerObject(int startX, int startY, int width, int height, IPOIAlgorithm* poiSource);
				static void matchObjects(std::vector<Poi>& iPts);
		};
	
	}
}


#endif
\end{lstlisting}
\end{framed}
\begin{framed}
\texttt{poi.h: Defines data structures for the points of interest used in algorithms like SURF or SIFT}
\begin{lstlisting}
#ifndef _POI_H_
#define _POI_H_

#ifndef CYTRUSALGLIB_API
	#ifdef CYTRUSALGLIB_EXPORTS
		#define CYTRUSALGLIB_API __declspec(dllexport)
	#else
		#define CYTRUSALGLIB_API __declspec(dllimport)
	#endif
#endif


class CYTRUSALGLIB_API Poi{
public:
	float x, y;
	float scale;
	float orientation;
	int laplacianSign;
	int descriptorSize;
	float* descriptor;
	float dx, dy;
	//int clusterIndex; - not used yet
	int matchesObjectNr;
	float matchedDistance;

	Poi(int descrSize=64) : orientation(0), descriptorSize(descrSize)
	{
		descriptor=new float[descriptorSize];
		matchesObjectNr=-1;
		matchedDistance=-1;
	}
};

#endif
\end{lstlisting}
\end{framed}
\begin{framed}
\texttt{SurfDescriptor.h : Describe POI's matrix.
Some parts are adapted from Bay's SURF implementation/notes:C. Evans, Research Into Robust Visual Features, MSc University of Bristol, 2008. }
\begin{lstlisting}
#ifndef _SURFDESCRIPTOR_H_
#define _SURFDESCRIPTOR_H_

#ifndef CYTRUSALGLIB_API
    #ifdef CYTRUSALGLIB_EXPORTS
        #define CYTRUSALGLIB_API __declspec(dllexport)
    #else
        #define CYTRUSALGLIB_API __declspec(dllimport)
    #endif
#endif

#define PI  3.1415926535897932384626433832795

#include <vector>
#include <iostream>
#include "math.h"
#include "poi.h"
#include "IDescriptor.h"

namespace cytrus{
    namespace alg{
		const float pi = float(PI);

        template <typename IntegralImageView>
        class SurfDescriptor:public IDescriptor{

            private:

				void getOrientation(Poi& pt);
				void getDescriptor(Poi& pt);

				inline float gaussian(int x, int y, float sig);
				inline float gaussian(float x, float y, float sig);
				inline float haarX(int row, int column, int s);
				inline float haarY(int row, int column, int s);
				float getAngleFromOrigin(float x, float y);

            protected:

            IntegralImageView* _img;
            int i_width, i_height;
			bool _oriented;

            public:
                SurfDescriptor(IntegralImageView& intImg, bool oriented=true);

				// Delayed image init constructor. Call setSourceIntegralImg to set image.
                SurfDescriptor(bool oriented=true);

				virtual ~SurfDescriptor(){}

                void setSourceIntegralImg(IntegralImageView& intImg);
                void setParameters(const int octaves, 
                                   const int intervals, 
                                   const int sampling , 
                                   const float threshold);
                
                virtual void getDescriptorsFor(std::vector<Poi>& iPts_out);
        };
    }
}

#include "SurfDescriptor.hpp"

#endif
\end{lstlisting}
\end{framed}
\begin{framed}
\texttt{SurfAlg.h : The implementation for the SurfAlgorithm}
\begin{lstlisting}
#ifndef _SURFALG_H_
#define _SURFALG_H_

#ifndef CYTRUSALGLIB_API
    #ifdef CYTRUSALGLIB_EXPORTS
        #define CYTRUSALGLIB_API __declspec(dllexport)
    #else
        #define CYTRUSALGLIB_API __declspec(dllimport)
    #endif
#endif

#include <vector>
#include "IImageConsumer.h"
#include "IImageSource.h"
#include "IPOIAlgorithm.h"
#include "ObjectPoiStorage.h"
#include "poi.h"

using namespace cytrus::cameraHAL;
using namespace cytrus::alg;

namespace cytrus{
    namespace alg{
        class CYTRUSALGLIB_API SurfAlg : public IPOIAlgorithm{
            private:
                volatile int _pWidth, _pHeight;
                volatile bool _isCustomPrelSize;
				static ObjectPoiStorage* _store;

            public:
                SurfAlg(IImageSource* imgSrc, POIAlgResult outputFunc, int index);
                ~SurfAlg();

                virtual void processImage(unsigned long dwSize, unsigned char* pbData);

                virtual bool setProcessingSize(int newWidth, int newHeight);
                
                //Actions to take when the image from the source changes size
                virtual void onSourceSizeChange();
        };
    }
}

#endif

\end{lstlisting}
\end{framed}
\begin{framed}
\texttt{CytrusManagedLib.h Assures .NET managed camera wrappers for the unmanaged cytrus code}
\begin{lstlisting}
#pragma once
#include "CytrusAlgLib.h"
#include <list>
#include <iostream>

using namespace System;
using namespace System::Collections::Generic;
using namespace System::Collections::ObjectModel;
using namespace System::Runtime::InteropServices;
using namespace System::Windows::Media;

using namespace cytrus::cameraHAL;
using namespace cytrus::alg;


namespace cytrus {
	namespace managed{

		delegate void RenderResultCallbackProc(int dwSize, unsigned char* pbData, int index);
		delegate void NewImageCallback();
		public delegate void ImageCaptureCallback(array<byte>^ pbData, List<Poi_m^>^ poiData);

		public ref struct OutputMode{
			String^ modeName;
			PixelFormat^ pixelFormat;

			virtual String^ ToString() override
			{
				return modeName;
			}
		};

		//public delegate void OutputModeCallback(OutputMode^ newMode); // outputMode change events (not used)

		// Managed class that is responsible for camera and capture management. It uses the unmanaged classes from cytrus::alg and cytrus::hal
		// for running the SURF algorithm in real time on the captured video.
		//
		// This class also implements coarse-grained parallelism, dispatching work on multiple threads (there are multiple
		// separate instances of the algorithm working in the same time)
		public ref class CameraMgr
		{	
		private:
			static GCHandle gch, nigch;
			static int thNr=0;
			//static int lastdwSize; // outputMode change events (not used)
			std::list<IPOIAlgorithm*>* alg_ProcessingPool;
			Dictionary<int, int>^ threadIndexes;

			RenderResultCallbackProc^ fPtr;
			NewImageCallback^ newImage;
			DirectShowCameraSource* cs;
			ObservableCollection<String^>^ cList;
			
			ObservableCollection<OutputMode^>^ outputModes;
			POIAlgResult result;
			NewImageAvailableCallback newImageAvailable;

			void callImageCaptureEvent(int dwSize, unsigned char* pbData, int index);
			void newImageAvailableEvent();
			void cameraNotifyConsumers(Object^ o);

		public:
			int _camWidth, _camHeight;

			// Treat this event to recieve the images from the selected camera
			// as an array of bytes.
			event ImageCaptureCallback^ onImageAvailableForRendering;
			//event OutputModeCallback^ onOutputModeChange; // outputMode change events (not used)
			
			CameraMgr();
			!CameraMgr();

			void selectCamera(int i);
			void refreshCameraList();
			ObservableCollection<String^>^ getCameraList();
			void showPropertiesDialog(IntPtr window);

			void setActiveOutputMode(int modeIndex);
			ObservableCollection<OutputMode^>^ getOutputModesList();
			bool setProcessingSize(int width, int height);

			void startCapture();
			void stopCapture();

		};
	}
}

\end{lstlisting}
\end{framed}
\begin{framed}
\texttt{ImageFileManaged.h Assures .NET managed wrappers for the image file unmanaged cytrus code}
\begin{lstlisting}
#pragma once
#include "CytrusAlgLib.h"
#include "CytrusManagedLib.h"

using namespace System::Runtime::InteropServices;
using namespace cytrus::cameraHAL;
using namespace cytrus::alg;


namespace cytrus {
	namespace managed{
	
		public ref class ImageFileMgr
		{	
		private:
			GCHandle gch;
			IPOIAlgorithm* alg;
			bool isInvalid;

			RenderResultCallbackProc^ fPtr;
			FileImageSource* fs;
			char* filePath;
			
			ObservableCollection<OutputMode^>^ outputModes;
			POIAlgResult result;
			NewImageAvailableCallback newImageAvailable;

			void callImageCaptureEvent(int dwSize, unsigned char* pbData, int index);

		public:

			property int _imgWidth
			{
				int get()
				{
					if(fs!=NULL)
						return fs->width;
					return 0;
				}
			}

			property int _imgHeight
			{
				int get()
				{
					if(fs!=NULL)
						return fs->height;
					return 0;
				}
			}

			// Treat this event to recieve the images from the selected camera
			// as an array of bytes.
			event ImageCaptureCallback^ onImageAvailableForRendering;
			
			ImageFileMgr();
			!ImageFileMgr();
			~ImageFileMgr();

			void setActiveOutputMode(int modeIndex);
			ObservableCollection<OutputMode^>^ getOutputModesList();
			bool setProcessingSize(int width, int height);
			void setImagePath(String^ path);

			void startImageProcessing();
			void freeResources();

			//returns the object index
			int registerObject(int startX, int startY, int width, int height);

		};
	
	}
}

\end{lstlisting}
\end{framed}
% ********** End of appendix **********
